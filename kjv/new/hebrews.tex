\book{The Epistle of Paul the Apostle to the Hebrews}


1:1 God, who at sundry times and in divers manners spake in time past
unto the fathers by the prophets, 1:2 Hath in these last days spoken
unto us by his Son, whom he hath appointed heir of all things, by whom
also he made the worlds; 1:3 Who being the brightness of his glory,
and the express image of his person, and upholding all things by the
word of his power, when he had by himself purged our sins, sat down on
the right hand of the Majesty on high: 1:4 Being made so much better
than the angels, as he hath by inheritance obtained a more excellent
name than they.

1:5 For unto which of the angels said he at any time, Thou art my Son,
this day have I begotten thee? And again, I will be to him a Father,
and he shall be to me a Son?  1:6 And again, when he bringeth in the
firstbegotten into the world, he saith, And let all the angels of God
worship him.

1:7 And of the angels he saith, Who maketh his angels spirits, and his
ministers a flame of fire.

1:8 But unto the Son he saith, Thy throne, O God, is for ever and
ever: a sceptre of righteousness is the sceptre of thy kingdom.

1:9 Thou hast loved righteousness, and hated iniquity; therefore God,
even thy God, hath anointed thee with the oil of gladness above thy
fellows.

1:10 And, Thou, Lord, in the beginning hast laid the foundation of the
earth; and the heavens are the works of thine hands: 1:11 They shall
perish; but thou remainest; and they all shall wax old as doth a
garment; 1:12 And as a vesture shalt thou fold them up, and they shall
be changed: but thou art the same, and thy years shall not fail.

1:13 But to which of the angels said he at any time, Sit on my right
hand, until I make thine enemies thy footstool?  1:14 Are they not all
ministering spirits, sent forth to minister for them who shall be
heirs of salvation?  2:1 Therefore we ought to give the more earnest
heed to the things which we have heard, lest at any time we should let
them slip.

2:2 For if the word spoken by angels was stedfast, and every
transgression and disobedience received a just recompence of reward;
2:3 How shall we escape, if we neglect so great salvation; which at
the first began to be spoken by the Lord, and was confirmed unto us by
them that heard him; 2:4 God also bearing them witness, both with
signs and wonders, and with divers miracles, and gifts of the Holy
Ghost, according to his own will?  2:5 For unto the angels hath he not
put in subjection the world to come, whereof we speak.

2:6 But one in a certain place testified, saying, What is man, that
thou art mindful of him? or the son of man that thou visitest him?
2:7 Thou madest him a little lower than the angels; thou crownedst him
with glory and honour, and didst set him over the works of thy hands:
2:8 Thou hast put all things in subjection under his feet. For in that
he put all in subjection under him, he left nothing that is not put
under him.

But now we see not yet all things put under him.

2:9 But we see Jesus, who was made a little lower than the angels for
the suffering of death, crowned with glory and honour; that he by the
grace of God should taste death for every man.

2:10 For it became him, for whom are all things, and by whom are all
things, in bringing many sons unto glory, to make the captain of their
salvation perfect through sufferings.

2:11 For both he that sanctifieth and they who are sanctified are all
of one: for which cause he is not ashamed to call them brethren, 2:12
Saying, I will declare thy name unto my brethren, in the midst of the
church will I sing praise unto thee.

2:13 And again, I will put my trust in him. And again, Behold I and
the children which God hath given me.

2:14 Forasmuch then as the children are partakers of flesh and blood,
he also himself likewise took part of the same; that through death he
might destroy him that had the power of death, that is, the devil;
2:15 And deliver them who through fear of death were all their
lifetime subject to bondage.

2:16 For verily he took not on him the nature of angels; but he took
on him the seed of Abraham.

2:17 Wherefore in all things it behoved him to be made like unto his
brethren, that he might be a merciful and faithful high priest in
things pertaining to God, to make reconciliation for the sins of the
people.

2:18 For in that he himself hath suffered being tempted, he is able to
succour them that are tempted.

3:1 Wherefore, holy brethren, partakers of the heavenly calling,
consider the Apostle and High Priest of our profession, Christ Jesus;
3:2 Who was faithful to him that appointed him, as also Moses was
faithful in all his house.

3:3 For this man was counted worthy of more glory than Moses, inasmuch
as he who hath builded the house hath more honour than the house.

3:4 For every house is builded by some man; but he that built all
things is God.

3:5 And Moses verily was faithful in all his house, as a servant, for
a testimony of those things which were to be spoken after; 3:6 But
Christ as a son over his own house; whose house are we, if we hold
fast the confidence and the rejoicing of the hope firm unto the end.

3:7 Wherefore (as the Holy Ghost saith, To day if ye will hear his
voice, 3:8 Harden not your hearts, as in the provocation, in the day
of temptation in the wilderness: 3:9 When your fathers tempted me,
proved me, and saw my works forty years.

3:10 Wherefore I was grieved with that generation, and said, They do
alway err in their heart; and they have not known my ways.

3:11 So I sware in my wrath, They shall not enter into my rest.)  3:12
Take heed, brethren, lest there be in any of you an evil heart of
unbelief, in departing from the living God.

3:13 But exhort one another daily, while it is called To day; lest any
of you be hardened through the deceitfulness of sin.

3:14 For we are made partakers of Christ, if we hold the beginning of
our confidence stedfast unto the end; 3:15 While it is said, To day if
ye will hear his voice, harden not your hearts, as in the provocation.

3:16 For some, when they had heard, did provoke: howbeit not all that
came out of Egypt by Moses.

3:17 But with whom was he grieved forty years? was it not with them
that had sinned, whose carcases fell in the wilderness?  3:18 And to
whom sware he that they should not enter into his rest, but to them
that believed not?  3:19 So we see that they could not enter in
because of unbelief.

4:1 Let us therefore fear, lest, a promise being left us of entering
into his rest, any of you should seem to come short of it.

4:2 For unto us was the gospel preached, as well as unto them: but the
word preached did not profit them, not being mixed with faith in them
that heard it.

4:3 For we which have believed do enter into rest, as he said, As I
have sworn in my wrath, if they shall enter into my rest: although the
works were finished from the foundation of the world.

4:4 For he spake in a certain place of the seventh day on this wise,
And God did rest the seventh day from all his works.

4:5 And in this place again, If they shall enter into my rest.

4:6 Seeing therefore it remaineth that some must enter therein, and
they to whom it was first preached entered not in because of unbelief:
4:7 Again, he limiteth a certain day, saying in David, To day, after
so long a time; as it is said, To day if ye will hear his voice,
harden not your hearts.

4:8 For if Jesus had given them rest, then would he not afterward have
spoken of another day.

4:9 There remaineth therefore a rest to the people of God.

4:10 For he that is entered into his rest, he also hath ceased from
his own works, as God did from his.

4:11 Let us labour therefore to enter into that rest, lest any man
fall after the same example of unbelief.

4:12 For the word of God is quick, and powerful, and sharper than any
twoedged sword, piercing even to the dividing asunder of soul and
spirit, and of the joints and marrow, and is a discerner of the
thoughts and intents of the heart.

4:13 Neither is there any creature that is not manifest in his sight:
but all things are naked and opened unto the eyes of him with whom we
have to do.

4:14 Seeing then that we have a great high priest, that is passed into
the heavens, Jesus the Son of God, let us hold fast our profession.

4:15 For we have not an high priest which cannot be touched with the
feeling of our infirmities; but was in all points tempted like as we
are, yet without sin.

4:16 Let us therefore come boldly unto the throne of grace, that we
may obtain mercy, and find grace to help in time of need.

5:1 For every high priest taken from among men is ordained for men in
things pertaining to God, that he may offer both gifts and sacrifices
for sins: 5:2 Who can have compassion on the ignorant, and on them
that are out of the way; for that he himself also is compassed with
infirmity.

5:3 And by reason hereof he ought, as for the people, so also for
himself, to offer for sins.

5:4 And no man taketh this honour unto himself, but he that is called
of God, as was Aaron.

5:5 So also Christ glorified not himself to be made an high priest;
but he that said unto him, Thou art my Son, to day have I begotten
thee.

5:6 As he saith also in another place, Thou art a priest for ever
after the order of Melchisedec.

5:7 Who in the days of his flesh, when he had offered up prayers and
supplications with strong crying and tears unto him that was able to
save him from death, and was heard in that he feared; 5:8 Though he
were a Son, yet learned he obedience by the things which he suffered;
5:9 And being made perfect, he became the author of eternal salvation
unto all them that obey him; 5:10 Called of God an high priest after
the order of Melchisedec.

5:11 Of whom we have many things to say, and hard to be uttered,
seeing ye are dull of hearing.

5:12 For when for the time ye ought to be teachers, ye have need that
one teach you again which be the first principles of the oracles of
God; and are become such as have need of milk, and not of strong meat.

5:13 For every one that useth milk is unskilful in the word of
righteousness: for he is a babe.

5:14 But strong meat belongeth to them that are of full age, even
those who by reason of use have their senses exercised to discern both
good and evil.

6:1 Therefore leaving the principles of the doctrine of Christ, let us
go on unto perfection; not laying again the foundation of repentance
from dead works, and of faith toward God, 6:2 Of the doctrine of
baptisms, and of laying on of hands, and of resurrection of the dead,
and of eternal judgment.

6:3 And this will we do, if God permit.

6:4 For it is impossible for those who were once enlightened, and have
tasted of the heavenly gift, and were made partakers of the Holy
Ghost, 6:5 And have tasted the good word of God, and the powers of the
world to come, 6:6 If they shall fall away, to renew them again unto
repentance; seeing they crucify to themselves the Son of God afresh,
and put him to an open shame.

6:7 For the earth which drinketh in the rain that cometh oft upon it,
and bringeth forth herbs meet for them by whom it is dressed,
receiveth blessing from God: 6:8 But that which beareth thorns and
briers is rejected, and is nigh unto cursing; whose end is to be
burned.

6:9 But, beloved, we are persuaded better things of you, and things
that accompany salvation, though we thus speak.

6:10 For God is not unrighteous to forget your work and labour of
love, which ye have shewed toward his name, in that ye have ministered
to the saints, and do minister.

6:11 And we desire that every one of you do shew the same diligence to
the full assurance of hope unto the end: 6:12 That ye be not slothful,
but followers of them who through faith and patience inherit the
promises.

6:13 For when God made promise to Abraham, because he could swear by
no greater, he sware by himself, 6:14 Saying, Surely blessing I will
bless thee, and multiplying I will multiply thee.

6:15 And so, after he had patiently endured, he obtained the promise.

6:16 For men verily swear by the greater: and an oath for confirmation
is to them an end of all strife.

6:17 Wherein God, willing more abundantly to shew unto the heirs of
promise the immutability of his counsel, confirmed it by an oath: 6:18
That by two immutable things, in which it was impossible for God to
lie, we might have a strong consolation, who have fled for refuge to
lay hold upon the hope set before us: 6:19 Which hope we have as an
anchor of the soul, both sure and stedfast, and which entereth into
that within the veil; 6:20 Whither the forerunner is for us entered,
even Jesus, made an high priest for ever after the order of
Melchisedec.

7:1 For this Melchisedec, king of Salem, priest of the most high God,
who met Abraham returning from the slaughter of the kings, and blessed
him; 7:2 To whom also Abraham gave a tenth part of all; first being by
interpretation King of righteousness, and after that also King of
Salem, which is, King of peace; 7:3 Without father, without mother,
without descent, having neither beginning of days, nor end of life;
but made like unto the Son of God; abideth a priest continually.

7:4 Now consider how great this man was, unto whom even the patriarch
Abraham gave the tenth of the spoils.

7:5 And verily they that are of the sons of Levi, who receive the
office of the priesthood, have a commandment to take tithes of the
people according to the law, that is, of their brethren, though they
come out of the loins of Abraham: 7:6 But he whose descent is not
counted from them received tithes of Abraham, and blessed him that had
the promises.

7:7 And without all contradiction the less is blessed of the better.

7:8 And here men that die receive tithes; but there he receiveth them,
of whom it is witnessed that he liveth.

7:9 And as I may so say, Levi also, who receiveth tithes, payed tithes
in Abraham.

7:10 For he was yet in the loins of his father, when Melchisedec met
him.

7:11 If therefore perfection were by the Levitical priesthood, (for
under it the people received the law,) what further need was there
that another priest should rise after the order of Melchisedec, and
not be called after the order of Aaron?  7:12 For the priesthood being
changed, there is made of necessity a change also of the law.

7:13 For he of whom these things are spoken pertaineth to another
tribe, of which no man gave attendance at the altar.

7:14 For it is evident that our Lord sprang out of Juda; of which
tribe Moses spake nothing concerning priesthood.

7:15 And it is yet far more evident: for that after the similitude of
Melchisedec there ariseth another priest, 7:16 Who is made, not after
the law of a carnal commandment, but after the power of an endless
life.

7:17 For he testifieth, Thou art a priest for ever after the order of
Melchisedec.

7:18 For there is verily a disannulling of the commandment going
before for the weakness and unprofitableness thereof.

7:19 For the law made nothing perfect, but the bringing in of a better
hope did; by the which we draw nigh unto God.

7:20 And inasmuch as not without an oath he was made priest: 7:21 (For
those priests were made without an oath; but this with an oath by him
that said unto him, The Lord sware and will not repent, Thou art a
priest for ever after the order of Melchisedec:) 7:22 By so much was
Jesus made a surety of a better testament.

7:23 And they truly were many priests, because they were not suffered
to continue by reason of death: 7:24 But this man, because he
continueth ever, hath an unchangeable priesthood.

7:25 Wherefore he is able also to save them to the uttermost that come
unto God by him, seeing he ever liveth to make intercession for them.

7:26 For such an high priest became us, who is holy, harmless,
undefiled, separate from sinners, and made higher than the heavens;
7:27 Who needeth not daily, as those high priests, to offer up
sacrifice, first for his own sins, and then for the people's: for this
he did once, when he offered up himself.

7:28 For the law maketh men high priests which have infirmity; but the
word of the oath, which was since the law, maketh the Son, who is
consecrated for evermore.

8:1 Now of the things which we have spoken this is the sum: We have
such an high priest, who is set on the right hand of the throne of the
Majesty in the heavens; 8:2 A minister of the sanctuary, and of the
true tabernacle, which the Lord pitched, and not man.

8:3 For every high priest is ordained to offer gifts and sacrifices:
wherefore it is of necessity that this man have somewhat also to
offer.

8:4 For if he were on earth, he should not be a priest, seeing that
there are priests that offer gifts according to the law: 8:5 Who serve
unto the example and shadow of heavenly things, as Moses was
admonished of God when he was about to make the tabernacle: for, See,
saith he, that thou make all things according to the pattern shewed to
thee in the mount.

8:6 But now hath he obtained a more excellent ministry, by how much
also he is the mediator of a better covenant, which was established
upon better promises.

8:7 For if that first covenant had been faultless, then should no
place have been sought for the second.

8:8 For finding fault with them, he saith, Behold, the days come,
saith the Lord, when I will make a new covenant with the house of
Israel and with the house of Judah: 8:9 Not according to the covenant
that I made with their fathers in the day when I took them by the hand
to lead them out of the land of Egypt; because they continued not in
my covenant, and I regarded them not, saith the Lord.

8:10 For this is the covenant that I will make with the house of
Israel after those days, saith the Lord; I will put my laws into their
mind, and write them in their hearts: and I will be to them a God, and
they shall be to me a people: 8:11 And they shall not teach every man
his neighbour, and every man his brother, saying, Know the Lord: for
all shall know me, from the least to the greatest.

8:12 For I will be merciful to their unrighteousness, and their sins
and their iniquities will I remember no more.

8:13 In that he saith, A new covenant, he hath made the first old. Now
that which decayeth and waxeth old is ready to vanish away.

9:1 Then verily the first covenant had also ordinances of divine
service, and a worldly sanctuary.

9:2 For there was a tabernacle made; the first, wherein was the
candlestick, and the table, and the shewbread; which is called the
sanctuary.

9:3 And after the second veil, the tabernacle which is called the
Holiest of all; 9:4 Which had the golden censer, and the ark of the
covenant overlaid round about with gold, wherein was the golden pot
that had manna, and Aaron's rod that budded, and the tables of the
covenant; 9:5 And over it the cherubims of glory shadowing the
mercyseat; of which we cannot now speak particularly.

9:6 Now when these things were thus ordained, the priests went always
into the first tabernacle, accomplishing the service of God.

9:7 But into the second went the high priest alone once every year,
not without blood, which he offered for himself, and for the errors of
the people: 9:8 The Holy Ghost this signifying, that the way into the
holiest of all was not yet made manifest, while as the first
tabernacle was yet standing: 9:9 Which was a figure for the time then
present, in which were offered both gifts and sacrifices, that could
not make him that did the service perfect, as pertaining to the
conscience; 9:10 Which stood only in meats and drinks, and divers
washings, and carnal ordinances, imposed on them until the time of
reformation.

9:11 But Christ being come an high priest of good things to come, by a
greater and more perfect tabernacle, not made with hands, that is to
say, not of this building; 9:12 Neither by the blood of goats and
calves, but by his own blood he entered in once into the holy place,
having obtained eternal redemption for us.

9:13 For if the blood of bulls and of goats, and the ashes of an
heifer sprinkling the unclean, sanctifieth to the purifying of the
flesh: 9:14 How much more shall the blood of Christ, who through the
eternal Spirit offered himself without spot to God, purge your
conscience from dead works to serve the living God?  9:15 And for this
cause he is the mediator of the new testament, that by means of death,
for the redemption of the transgressions that were under the first
testament, they which are called might receive the promise of eternal
inheritance.

9:16 For where a testament is, there must also of necessity be the
death of the testator.

9:17 For a testament is of force after men are dead: otherwise it is
of no strength at all while the testator liveth.

9:18 Whereupon neither the first testament was dedicated without
blood.

9:19 For when Moses had spoken every precept to all the people
according to the law, he took the blood of calves and of goats, with
water, and scarlet wool, and hyssop, and sprinkled both the book, and
all the people, 9:20 Saying, This is the blood of the testament which
God hath enjoined unto you.

9:21 Moreover he sprinkled with blood both the tabernacle, and all the
vessels of the ministry.

9:22 And almost all things are by the law purged with blood; and
without shedding of blood is no remission.

9:23 It was therefore necessary that the patterns of things in the
heavens should be purified with these; but the heavenly things
themselves with better sacrifices than these.

9:24 For Christ is not entered into the holy places made with hands,
which are the figures of the true; but into heaven itself, now to
appear in the presence of God for us: 9:25 Nor yet that he should
offer himself often, as the high priest entereth into the holy place
every year with blood of others; 9:26 For then must he often have
suffered since the foundation of the world: but now once in the end of
the world hath he appeared to put away sin by the sacrifice of
himself.

9:27 And as it is appointed unto men once to die, but after this the
judgment: 9:28 So Christ was once offered to bear the sins of many;
and unto them that look for him shall he appear the second time
without sin unto salvation.

10:1 For the law having a shadow of good things to come, and not the
very image of the things, can never with those sacrifices which they
offered year by year continually make the comers thereunto perfect.

10:2 For then would they not have ceased to be offered? because that
the worshippers once purged should have had no more conscience of
sins.

10:3 But in those sacrifices there is a remembrance again made of sins
every year.

10:4 For it is not possible that the blood of bulls and of goats
should take away sins.

10:5 Wherefore when he cometh into the world, he saith, Sacrifice and
offering thou wouldest not, but a body hast thou prepared me: 10:6 In
burnt offerings and sacrifices for sin thou hast had no pleasure.

10:7 Then said I, Lo, I come (in the volume of the book it is written
of me,) to do thy will, O God.

10:8 Above when he said, Sacrifice and offering and burnt offerings
and offering for sin thou wouldest not, neither hadst pleasure
therein; which are offered by the law; 10:9 Then said he, Lo, I come
to do thy will, O God. He taketh away the first, that he may establish
the second.

10:10 By the which will we are sanctified through the offering of the
body of Jesus Christ once for all.

10:11 And every priest standeth daily ministering and offering
oftentimes the same sacrifices, which can never take away sins: 10:12
But this man, after he had offered one sacrifice for sins for ever,
sat down on the right hand of God; 10:13 From henceforth expecting
till his enemies be made his footstool.

10:14 For by one offering he hath perfected for ever them that are
sanctified.

10:15 Whereof the Holy Ghost also is a witness to us: for after that
he had said before, 10:16 This is the covenant that I will make with
them after those days, saith the Lord, I will put my laws into their
hearts, and in their minds will I write them; 10:17 And their sins and
iniquities will I remember no more.

10:18 Now where remission of these is, there is no more offering for
sin.

10:19 Having therefore, brethren, boldness to enter into the holiest
by the blood of Jesus, 10:20 By a new and living way, which he hath
consecrated for us, through the veil, that is to say, his flesh; 10:21
And having an high priest over the house of God; 10:22 Let us draw
near with a true heart in full assurance of faith, having our hearts
sprinkled from an evil conscience, and our bodies washed with pure
water.

10:23 Let us hold fast the profession of our faith without wavering;
(for he is faithful that promised;) 10:24 And let us consider one
another to provoke unto love and to good works: 10:25 Not forsaking
the assembling of ourselves together, as the manner of some is; but
exhorting one another: and so much the more, as ye see the day
approaching.

10:26 For if we sin wilfully after that we have received the knowledge
of the truth, there remaineth no more sacrifice for sins, 10:27 But a
certain fearful looking for of judgment and fiery indignation, which
shall devour the adversaries.

10:28 He that despised Moses' law died without mercy under two or
three witnesses: 10:29 Of how much sorer punishment, suppose ye, shall
he be thought worthy, who hath trodden under foot the Son of God, and
hath counted the blood of the covenant, wherewith he was sanctified,
an unholy thing, and hath done despite unto the Spirit of grace?
10:30 For we know him that hath said, Vengeance belongeth unto me, I
will recompense, saith the Lord. And again, The Lord shall judge his
people.

10:31 It is a fearful thing to fall into the hands of the living God.

10:32 But call to remembrance the former days, in which, after ye were
illuminated, ye endured a great fight of afflictions; 10:33 Partly,
whilst ye were made a gazingstock both by reproaches and afflictions;
and partly, whilst ye became companions of them that were so used.

10:34 For ye had compassion of me in my bonds, and took joyfully the
spoiling of your goods, knowing in yourselves that ye have in heaven a
better and an enduring substance.

10:35 Cast not away therefore your confidence, which hath great
recompence of reward.

10:36 For ye have need of patience, that, after ye have done the will
of God, ye might receive the promise.

10:37 For yet a little while, and he that shall come will come, and
will not tarry.

10:38 Now the just shall live by faith: but if any man draw back, my
soul shall have no pleasure in him.

10:39 But we are not of them who draw back unto perdition; but of them
that believe to the saving of the soul.

11:1 Now faith is the substance of things hoped for, the evidence of
things not seen.

11:2 For by it the elders obtained a good report.

11:3 Through faith we understand that the worlds were framed by the
word of God, so that things which are seen were not made of things
which do appear.

11:4 By faith Abel offered unto God a more excellent sacrifice than
Cain, by which he obtained witness that he was righteous, God
testifying of his gifts: and by it he being dead yet speaketh.

11:5 By faith Enoch was translated that he should not see death; and
was not found, because God had translated him: for before his
translation he had this testimony, that he pleased God.

11:6 But without faith it is impossible to please him: for he that
cometh to God must believe that he is, and that he is a rewarder of
them that diligently seek him.

11:7 By faith Noah, being warned of God of things not seen as yet,
moved with fear, prepared an ark to the saving of his house; by the
which he condemned the world, and became heir of the righteousness
which is by faith.

11:8 By faith Abraham, when he was called to go out into a place which
he should after receive for an inheritance, obeyed; and he went out,
not knowing whither he went.

11:9 By faith he sojourned in the land of promise, as in a strange
country, dwelling in tabernacles with Isaac and Jacob, the heirs with
him of the same promise: 11:10 For he looked for a city which hath
foundations, whose builder and maker is God.

11:11 Through faith also Sara herself received strength to conceive
seed, and was delivered of a child when she was past age, because she
judged him faithful who had promised.

11:12 Therefore sprang there even of one, and him as good as dead, so
many as the stars of the sky in multitude, and as the sand which is by
the sea shore innumerable.

11:13 These all died in faith, not having received the promises, but
having seen them afar off, and were persuaded of them, and embraced
them, and confessed that they were strangers and pilgrims on the
earth.

11:14 For they that say such things declare plainly that they seek a
country.

11:15 And truly, if they had been mindful of that country from whence
they came out, they might have had opportunity to have returned.

11:16 But now they desire a better country, that is, an heavenly:
wherefore God is not ashamed to be called their God: for he hath
prepared for them a city.

11:17 By faith Abraham, when he was tried, offered up Isaac: and he
that had received the promises offered up his only begotten son, 11:18
Of whom it was said, That in Isaac shall thy seed be called: 11:19
Accounting that God was able to raise him up, even from the dead; from
whence also he received him in a figure.

11:20 By faith Isaac blessed Jacob and Esau concerning things to come.

11:21 By faith Jacob, when he was a dying, blessed both the sons of
Joseph; and worshipped, leaning upon the top of his staff.

11:22 By faith Joseph, when he died, made mention of the departing of
the children of Israel; and gave commandment concerning his bones.

11:23 By faith Moses, when he was born, was hid three months of his
parents, because they saw he was a proper child; and they were not
afraid of the king's commandment.

11:24 By faith Moses, when he was come to years, refused to be called
the son of Pharaoh's daughter; 11:25 Choosing rather to suffer
affliction with the people of God, than to enjoy the pleasures of sin
for a season; 11:26 Esteeming the reproach of Christ greater riches
than the treasures in Egypt: for he had respect unto the recompence of
the reward.

11:27 By faith he forsook Egypt, not fearing the wrath of the king:
for he endured, as seeing him who is invisible.

11:28 Through faith he kept the passover, and the sprinkling of blood,
lest he that destroyed the firstborn should touch them.

11:29 By faith they passed through the Red sea as by dry land: which
the Egyptians assaying to do were drowned.

11:30 By faith the walls of Jericho fell down, after they were
compassed about seven days.

11:31 By faith the harlot Rahab perished not with them that believed
not, when she had received the spies with peace.

11:32 And what shall I more say? for the time would fail me to tell of
Gedeon, and of Barak, and of Samson, and of Jephthae; of David also,
and Samuel, and of the prophets: 11:33 Who through faith subdued
kingdoms, wrought righteousness, obtained promises, stopped the mouths
of lions.

11:34 Quenched the violence of fire, escaped the edge of the sword,
out of weakness were made strong, waxed valiant in fight, turned to
flight the armies of the aliens.

11:35 Women received their dead raised to life again: and others were
tortured, not accepting deliverance; that they might obtain a better
resurrection: 11:36 And others had trial of cruel mockings and
scourgings, yea, moreover of bonds and imprisonment: 11:37 They were
stoned, they were sawn asunder, were tempted, were slain with the
sword: they wandered about in sheepskins and goatskins; being
destitute, afflicted, tormented; 11:38 (Of whom the world was not
worthy:) they wandered in deserts, and in mountains, and in dens and
caves of the earth.

11:39 And these all, having obtained a good report through faith,
received not the promise: 11:40 God having provided some better thing
for us, that they without us should not be made perfect.

12:1 Wherefore seeing we also are compassed about with so great a
cloud of witnesses, let us lay aside every weight, and the sin which
doth so easily beset us, and let us run with patience the race that is
set before us, 12:2 Looking unto Jesus the author and finisher of our
faith; who for the joy that was set before him endured the cross,
despising the shame, and is set down at the right hand of the throne
of God.

12:3 For consider him that endured such contradiction of sinners
against himself, lest ye be wearied and faint in your minds.

12:4 Ye have not yet resisted unto blood, striving against sin.

12:5 And ye have forgotten the exhortation which speaketh unto you as
unto children, My son, despise not thou the chastening of the Lord,
nor faint when thou art rebuked of him: 12:6 For whom the Lord loveth
he chasteneth, and scourgeth every son whom he receiveth.

12:7 If ye endure chastening, God dealeth with you as with sons; for
what son is he whom the father chasteneth not?  12:8 But if ye be
without chastisement, whereof all are partakers, then are ye bastards,
and not sons.

12:9 Furthermore we have had fathers of our flesh which corrected us,
and we gave them reverence: shall we not much rather be in subjection
unto the Father of spirits, and live?  12:10 For they verily for a few
days chastened us after their own pleasure; but he for our profit,
that we might be partakers of his holiness.

12:11 Now no chastening for the present seemeth to be joyous, but
grievous: nevertheless afterward it yieldeth the peaceable fruit of
righteousness unto them which are exercised thereby.

12:12 Wherefore lift up the hands which hang down, and the feeble
knees; 12:13 And make straight paths for your feet, lest that which is
lame be turned out of the way; but let it rather be healed.

12:14 Follow peace with all men, and holiness, without which no man
shall see the Lord: 12:15 Looking diligently lest any man fail of the
grace of God; lest any root of bitterness springing up trouble you,
and thereby many be defiled; 12:16 Lest there be any fornicator, or
profane person, as Esau, who for one morsel of meat sold his
birthright.

12:17 For ye know how that afterward, when he would have inherited the
blessing, he was rejected: for he found no place of repentance, though
he sought it carefully with tears.

12:18 For ye are not come unto the mount that might be touched, and
that burned with fire, nor unto blackness, and darkness, and tempest,
12:19 And the sound of a trumpet, and the voice of words; which voice
they that heard intreated that the word should not be spoken to them
any more: 12:20 (For they could not endure that which was commanded,
And if so much as a beast touch the mountain, it shall be stoned, or
thrust through with a dart: 12:21 And so terrible was the sight, that
Moses said, I exceedingly fear and quake:) 12:22 But ye are come unto
mount Sion, and unto the city of the living God, the heavenly
Jerusalem, and to an innumerable company of angels, 12:23 To the
general assembly and church of the firstborn, which are written in
heaven, and to God the Judge of all, and to the spirits of just men
made perfect, 12:24 And to Jesus the mediator of the new covenant, and
to the blood of sprinkling, that speaketh better things than that of
Abel.

12:25 See that ye refuse not him that speaketh. For if they escaped
not who refused him that spake on earth, much more shall not we
escape, if we turn away from him that speaketh from heaven: 12:26
Whose voice then shook the earth: but now he hath promised, saying,
Yet once more I shake not the earth only, but also heaven.

12:27 And this word, Yet once more, signifieth the removing of those
things that are shaken, as of things that are made, that those things
which cannot be shaken may remain.

12:28 Wherefore we receiving a kingdom which cannot be moved, let us
have grace, whereby we may serve God acceptably with reverence and
godly fear: 12:29 For our God is a consuming fire.

13:1 Let brotherly love continue.

13:2 Be not forgetful to entertain strangers: for thereby some have
entertained angels unawares.

13:3 Remember them that are in bonds, as bound with them; and them
which suffer adversity, as being yourselves also in the body.

13:4 Marriage is honourable in all, and the bed undefiled: but
whoremongers and adulterers God will judge.

13:5 Let your conversation be without covetousness; and be content
with such things as ye have: for he hath said, I will never leave
thee, nor forsake thee.

13:6 So that we may boldly say, The Lord is my helper, and I will not
fear what man shall do unto me.

13:7 Remember them which have the rule over you, who have spoken unto
you the word of God: whose faith follow, considering the end of their
conversation.

13:8 Jesus Christ the same yesterday, and to day, and for ever.

13:9 Be not carried about with divers and strange doctrines. For it is
a good thing that the heart be established with grace; not with meats,
which have not profited them that have been occupied therein.

13:10 We have an altar, whereof they have no right to eat which serve
the tabernacle.

13:11 For the bodies of those beasts, whose blood is brought into the
sanctuary by the high priest for sin, are burned without the camp.

13:12 Wherefore Jesus also, that he might sanctify the people with his
own blood, suffered without the gate.

13:13 Let us go forth therefore unto him without the camp, bearing his
reproach.

13:14 For here have we no continuing city, but we seek one to come.

13:15 By him therefore let us offer the sacrifice of praise to God
continually, that is, the fruit of our lips giving thanks to his name.

13:16 But to do good and to communicate forget not: for with such
sacrifices God is well pleased.

13:17 Obey them that have the rule over you, and submit yourselves:
for they watch for your souls, as they that must give account, that
they may do it with joy, and not with grief: for that is unprofitable
for you.

13:18 Pray for us: for we trust we have a good conscience, in all
things willing to live honestly.

13:19 But I beseech you the rather to do this, that I may be restored
to you the sooner.

13:20 Now the God of peace, that brought again from the dead our Lord
Jesus, that great shepherd of the sheep, through the blood of the
everlasting covenant, 13:21 Make you perfect in every good work to do
his will, working in you that which is wellpleasing in his sight,
through Jesus Christ; to whom be glory for ever and ever. Amen.

13:22 And I beseech you, brethren, suffer the word of exhortation: for
I have written a letter unto you in few words.

13:23 Know ye that our brother Timothy is set at liberty; with whom,
if he come shortly, I will see you.

13:24 Salute all them that have the rule over you, and all the saints.

They of Italy salute you.

13:25 Grace be with you all. Amen.

%%% Local Variables: 
%%% mode: latex
%%% TeX-master: "bible.tex"
%%% End: 
